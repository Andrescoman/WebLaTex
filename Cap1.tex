\chapter{Introducción}
\hrule \bigskip \vspace*{1cm}
%------------------------------------------------------------------------


El estrés es una respuesta natural del cuerpo ante situaciones que percibimos como amenazantes o desafiantes. Esta reacción puede ser de corta duración, conocida como \textbf{estrés agudo}, o prolongarse en el tiempo, dando lugar al \textbf{estrés crónico}. Ambos tipos de estrés tienen efectos significativos en nuestra salud física y mental, aunque sus manifestaciones y consecuencias pueden variar considerablemente.

El estrés agudo se caracteriza por ser una respuesta inmediata y de corta duración ante un estímulo específico. Es el tipo de estrés que sentimos, por ejemplo, antes de un examen importante o durante una situación de emergencia. Aunque puede ser intenso, el estrés agudo suele desaparecer una vez que la situación estresante ha pasado. Este tipo de estrés puede incluso tener efectos positivos, como mejorar el rendimiento y la concentración en situaciones críticas (Smith, 2020).

Por otro lado, el estrés crónico se desarrolla cuando una persona está expuesta a factores estresantes de manera continua o repetitiva. Este tipo de estrés puede tener efectos perjudiciales a largo plazo, contribuyendo al desarrollo de diversas enfermedades, como trastornos cardiovasculares, problemas digestivos y trastornos de ansiedad (Jones \& Brown, 2019). El estrés crónico puede erosionar la salud y el bienestar general, afectando tanto el cuerpo como la mente.

%A lo largo de este artículo, exploraremos en detalle las características y las señales . Finalmente, nos centraremos en la detección del estrés agudo, analizando sus mecanismos, efectos inmediatos y estrategias para su manejo efectivo.

A lo largo de este artículo, exploraremos  las características, causas y consecuencias del estrés agudo y su adquisición mediante la señal EDA. Finalmente, nos centraremos en la detección del estrés agudo utilizando inteligencia artificial, analizando  la señal EDA  y poder  identificar y gestionar eficazmente el estrés %en tiempo real, proporcionando nuevas herramientas para el bienestar y la salud mental.
%%%%%%%
agudo proporcionando nuevas herramientas para el bienestar y la salud mental.




%A pesar de su prevalencia y consecuencias potenciales, el estrés agudo sigue siendo un fenómeno mal entendido. Mientras que el estrés crónico ha recibido una atención significativa en los últimos años, el estrés agudo ha sido ampliamente ignorado, dejando una brecha significativa en nuestra comprensión de sus causas, consecuencias y estrategias de manejo (Harris et al., 2016).



%El estrés agudo es un fenómeno  que afecta a individuos de todas las edades y condiciones. Definido como una respuesta temporal e intensa a una amenaza o peligro percibido (Lupien et al., 2009), el estrés agudo es una respuesta adaptativa normal diseñada para ayudar a los individuos a enfrentar desafíos inmediatos. Sin embargo, cuando el estrés agudo se vuelve excesivo o crónico, puede tener consecuencias devastadoras en la salud física y mental, incluyendo ansiedad, depresión, enfermedad cardiovascular y función cognitiva deteriorada (Kabat-Zinn, 2003; McEwen, 2007).

\section{Justificación}
% ¿ Por qué vale la pena buscar lograr el objetivo planteado?  Se explica los detalles detrás de que la pregunta planteada aún no ha sido respondida, por ejemplo: citar que nadie presentó una solución satisfactoria o que existen soluciones contradictorias.


La detección de estrés agudo es un área de investigación importante y relevante por varias razones:

El estrés en el lugar de trabajo puede afectar significativamente el bienestar del trabajador. Puede llevar a un bajo estado de ánimo y disminución de la atención, lo que puede afectar la calidad de vida del trabajador.

\begin{itemize}
  \item \textbf{Productividad de la empresa:} El estrés no solo afecta al individuo, sino también a la productividad de la empresa. Un trabajador fatigado puede tener un rendimiento reducido, lo que puede afectar la eficiencia y la rentabilidad de la empresa.
  \item Seguridad: El estrés  puede aumentar el riesgo de errores y accidentes. En el caso de los conductores, por ejemplo, la fatiga que deriva en estrés  es una de las principales causas de accidentes de tránsito.
  \item  \textbf{En el campo de la salud:} La detección de estrés agudo es especialmente relevante. Por ejemplo, en pacientes con cáncer, el estrés es un síntoma común que puede afectar en gran medida su vida diaria. Sin embargo, a menudo no se identifica ni se valora adecuadamente, lo que puede resultar en una falta de cuidado y tratamiento adecuados.
  \item \textbf{Investigación en salud:} Por lo tanto, la investigación en la detección de estrés agudo puede tener implicaciones significativas en diversas áreas, desde mejorar la salud y el bienestar de los individuos hasta aumentar la seguridad y la productividad en el lugar de trabajo. Además, puede contribuir a la mejora de las intervenciones y tratamientos en el campo de la salud. Por estas razones, justificar un trabajo de investigación en la detección de fatiga es tanto relevante como necesario.

\end{itemize}


\section{Relevancia y/o Motivación de la Propuesta}




El estrés agudo  se caracteriza como una forma de agotamiento objetivo y subjetivo que surge de la participación prolongada en actividades cognitivas \cite{ishii2014neural}. Tiene implicaciones \cite{grillon2015mental} , \cite{brown2019effects} , \cite{van2017effects} , \cite{dogan2023multi} , \cite{cropanzano2003relationship} sobre las emociones, el comportamiento, el bienestar físico y las interacciones sociales. Los efectos abarcan un espectro de emociones, que incluyen rabia y melancolía , así como una renuencia a participar en entornos sociales.

El estrés agudo es uno de los problemas más comunes que ocurren entre los pacientes \cite{martin2018mental},\cite{marcora2009mental}, en el que un individuo experimenta resistencia a una actividad\cite{meijman2000theory},mal  Humor\cite{hockey1983stress}  letargo \cite{martin2018mental}.
El estrés  se relaciona principalmente con la enfermedad, el envejecimiento
y la depresión.\cite{martin2018mental}.
%La fatiga mental  es un problema que afecta a la sociedad tecnificada hoy en dia .
%\cite{fatiga}

En situaciones de estrés agudo , se observan modificaciones significativas en las señales físicas, como el ritmo cardíaco, la respiración, la sudoración y la dilatación de las pupilas.
En estas situaciones  los dispositivos portátiles se pueden aprovechar para capturar las señales a través  de     sus  sensores, uno de ellos es la pulsera E4 wristband que mide  diversas señales entre ellas EDA ,BVP Figura \ref{fig:relo}.

%La  información que recoje los  dispositivos portátiles son  útiles para ayudar a  detectar fatiga obtener datos valiosos, como los signos fisiológicos de la persona.


%Investigaciones previas han evidenciado que la actividad electrodérmica (EDA), también conocida como respuesta galvánica de la piel (GSR), puede ser indicativa de fatiga mental.EDA registra los cambios en la sudoración al detectar las alteraciones en la conductividad eléctrica de la piel. 

%Aquí se escribe la introducción al problema, yendo de lo más general a lo más específico.

%De preferencia debería iniciarse con la motivación que nos llevó a escoger el problema de la tesis y explicando porque es un problema interesante de resolver.

%En seguida, una vez que se establece claramente cuál es la motivación para resolver el problema, se debe situar el mismo dentro de un contexto. Aquí es donde se debe indicar en que area de la ciencia de la computación está ubicado el problema.

%\textit{
%\textbf{Ejemplo:} Hoy en día debido a la gran cantidad de información generada cotidianamente, y a la caída del costo de los medios de almacenamiento, contamos con grandes bases de datos en muchas empresas (gigabytes o incluso terabytes de datos) ....
%}



La señal electrodérmica de la piel (EDA) mide  los cambios eléctricos en la superficie de la piel, que surgen cuando la piel recibe señales inervantes del cerebro. Para la mayoría de las personas, si experimentan activación emocional, aumento de la carga cognitiva o esfuerzo físico, su cerebro enviara señales a la piel para aumentar el nivel de sudoración ,con ello su posterior registro por medio de los sensores del EDA.
Las unidades de medida de la conductancia son microSiemens.



Las causas del estrés agudo pueden ser variadas, pero algunas de las más comunes incluyen:

\begin{itemize}


  \item \textbf{Exceso de trabajo intelectual:} Las personas que tienen un exceso de empleo de tipo intelectual, donde se les exige comprender, razonar, solucionar problemas, estar concentrados y memorizar de forma casi constante, son propensas a la fatiga mental.

  \item \textbf{Ritmo de vida frenético:} El ritmo de vida acelerado, especialmente en las ciudades, puede llevar al estrés mental.

  \item \textbf{Dificultades para gestionar el tiempo y para desconectar:} La incapacidad para manejar adecuadamente el tiempo y desconectar.


\end{itemize}
%También  se ha estudiado que el estres mental disminuye la conciencia situacional,  aspecto fundamental de la seguridad humana y laboral \cite{kelly2019analysis}.%porque ? %cual son las concecuencia ? 








%¿cual es el ámbito en que esto se desenvuelve? y ¿que necesidad
%existe para motivar una investigación en tu tema?
%Se explica cómo la investigación contribuirá con algo valioso ya sea a nivel científico o social, el impacto que puede tener y por qué es importante que alguien se tome el tiempo para trabajar en esa problemática.


%\section{Pregunta de Investigación}

%Es una pregunta que todavía no fue respondida satisfactoriamente. Debe ser clara, concisa, específica, neutral y enfocada. También deben ser lo suficientemente compleja como para que la pregunta requiera algo más que una respuesta de "sí" o "no", por lo que  usualmente comienza con: ¿Qué … ? ¿Cómo… ?






%\section{Hipótesis}
%La hipótesis es una afirmación de la cual no se sabe si es verdadera o falsa. Es la posible respuesta a la pregunta de investigación. El trabajo de investigación consiste en probar la veracidad de la hipótesis. Es el corazón de la investigación, por lo que una buena hipótesis con evidencia de efectividad debe ser buscada.


\subsection{Definición del Problema}

%Una vez que se indicó la motivación para resolver el problema y fue situado en su respectivo contexto, es necesario dejar claro cuál es el problema que se quiere resolver. Esta sección no debe ser muy larga, ya que se supone que en la sección anterior se hizo toda la introducción necesaria. Idealmente debe ser una sola oración que resuma claramente el problema que se quiere resolver.

%\textit{
%\textbf{Ejemplo:} Se necesita contar con nuevas técnicas que permitan tratar más eficientemente grandes cantidades de información para sistemas computacionales de todo tipo.
%}
%El estres es un que aceja las sosciedad , hoy endai hay dispositivos portatiles que almacena informacion util que ayudan a tener infromacoin valiosa como los signo fisologicos de la prsona  , es por eso que se encesita detectarlo lo mejor posible  posible para obtner patrones que describan su naturaleza 



%A medida que avanza la tecnología  también aumenta la fatiga  , \cite{fatiga} 

%Suele aparecer en situaciones de mucha presión y estrés psicológico, emocional o intelectual. Este cansancio mental  no solo se manifiesta en los pensamientos y la mente, sino que también puede repercutir directamente en el  cuerpo, manifestándose en forma de debilidad física,dolores de cabeza ,dificultar la calidad de sueño\cite{Hockey_2013} ,  u otras complicaciones.

%existen varias limitaciones

Existen varias  limitaciones con respecto a la detección de  estrés agudo.

%Por ejemplo enla validación de  los datos   , los estudios que se  realizaron fueron principalmente en entornos de laboratorio, pero las conclusiones  basadas en laboratorio no necesariamente se aplican a la evaluación  ambulatoria.

Por ejemplo ,en un estudio de laboratorio se  encontró que el 22\% de
los datos de actividad electrodérmica (EDA) recopilados por un dispositivo
portátil eran inutilizables (van Lier et al., 2020), mientras que un estudio
ambulatorio que utilizó el mismo dispositivo estimó el 78\% (van
Lier et al., 2020).

los modelos genericos random forest

%Además  existe poco  estudio donde  se haga un análisis exploratorio de los datos  para visualizar y encontrar  similitudes y diferencias ,dado que  los datos pertenecen a  diferentes  protocolos de estudio como se ha  visto anteriormente.

%es  un estado del organismo que aparece como consecuencia del estrés o de una época puntual de tensión\cite{kelly2019analysis}. 

%Sin embargo,esta poco claro  sus síntomas van más allá y su aparición puede dificultar la calidad del sueño, conllevar problemas musculares y dolores de cabeza



%Actualmente existe trabajos de  investigación que se centran en el  reconocimiento de estrés y a menudo gira en torno al análisis de información fisiológica\cite{dogan2022stress}, \cite{derdiyok2023biosignal}  \cite{parekh2020fatigue}. 


Debido a que el estrés es subjetivo y se expresa de manera diferente de una
persona a otra, los modelos genéricos de predicción del estrés tienen bajo rendimiento. Sólo los específicos
de una persona (dependientes del usuario) producen predicciones confiables, pero no son adaptables y su
implementación en entornos del mundo real es costosa.
Así como  en un entorno de oficina, un modelo  dependiente del usuario requeriría recopilar nuevos datos y entrenar un nuevo modelo para cada empleado. Además, una vez implementados, los modelos se deteriorarían y necesitarían
costosas actualizaciones periódicas porque el estrés es dinámico y depende de factores que no se tienen en cuenta .
%imprevisibles 

%cuál de los enfoques: (dependientes del usuario y los independientes del usuario )sobre la  señal EDA de baja resolución son más precisos para detectar patrones de fatiga.

%Existen mucho modelos de detección de estrés ,usando diversas resoluciones (tasa de frecuencia de muestreo ) Por lo tanto  pero es poco sabido  cual de las dos utiliar   ,este trabajo ara un analisis estadístico de modelos  de  detección de estres de  comparando diversas fuentes de datos 


%Por lo que  el monitorio  podría ayudar a minimizar cualquier consecuencia negativa.

%aca voy a a ser saltar la relevancia del tema porque es importante 















\section{Objetivos}
% ¿Que pretendes obtener o resolver? y en los objetivos específicos
% detallar cada una de las actividades que realizaras, en este punto debes
% ser muy exacto y concreto(recuerda que en base a esto justificaras % si estas obteniendo resultados)

\subsection{Objetivo General}
%El objetivo debe ser directamente verificable al final del trabajo de investigación, un buen objetivo demostrará que la hipótesis que está siendo probada es verdadera o no. Tenga en cuenta que el objetivo no debe ser confundido con el tema de investigación. El objetivo general y los objetivos específicos deben ser formulados de manera que se puedan verificar al final del trabajo, comienzan con verbos en infinitivo, por ejemplo: demostrar, mejorar, analizar.

Diseñar   un sistema para el análisis y la detección de estrés agudo  usando el modelo Random Forest y Perceptrón Multicapa  a partir de la actividad electrodermica de la piel de baja resolución .

Diseñar un modelo  un modelo que mejore la precisión de de los modelos genéricos

\subsection{Objetivos específicos}
%Los objetivos específicos no son etapas de la investigación, son subproductos que serán el resultado del proceso de verificación del objetivo general. Los objetivos específicos deben indicar una contribución.

\textit{
  %\textbf{Ejemplo:}
}

\begin{itemize}

  \item \textit{Preprocesar los datos  }

        %debido a las diferencias en las respuestas fisiológicas de los individuos. }

  \item \textit{Entrenar el modelo utilizando  Random Forest .}

  \item \textit{Identificar   características dominantes  }
        %utilizando la tecnica método One-Leave-Out }
  \item  Comparar los dos  enfoques  utilizando señales EDA de baja resolución.

        %\item Evaluar los resultados del análisis aplicando Cross-Validation 

  \item \textit{Validar la precisión del sistema }

\end{itemize}







%\section{Organización de la tesis}

% Una breve descripción de cada uno de los capítulos que estas
% desarrollando desde el CAP 2 hasta el capitulo antes del apéndice.
