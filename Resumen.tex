\chapter*{Resumen}
\hrule \bigskip \vspace*{1cm}


El uso de aplicaciones es una parte fundamental de la vida ,sin embargo su uso excesivo , trae como consecuencias estilos de
vida poco saludables y puede desencadenar en diversas problemas como problemas al corazón , estrés,fatiga etc. Si bien se han realizado
muchas investigaciones utilizando la señal Electrodermica de la piel(EDA) para detectar estrés ,existen pocos trabajos que realicen un
análisis estadístico sobre los modelos dependiente y independiente del usuario, en términos de demostrar que los primeros son mas precisos
, y no ocurre por casualidad en lo que a estrés concierne . Este trabajo , se centra en realizar un análisis sobre la capacidad de modelos dependientes e independientes del usuario usando   Random Forest  así como proponer un  sistema de detección de estrés  ,para
ello se utilizara la base de datos WESAD disponibles públicamente que contienen diferentes señales fisiológicas registradas con la pulsera Empatica E4.